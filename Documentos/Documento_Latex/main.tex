%Preamble
\documentclass[11pt,letterpaper,twocolumn]{article}
%\brokenpenalty=10000 
%\hyphenpenalty=5000 
\usepackage[spanish]{babel}
\usepackage[utf8]{inputenc}
\usepackage{times}
\usepackage[backend=biber,style=ieee]{biblatex}
\usepackage[T1]{fontenc}
\usepackage{cancel}
\usepackage{tabularx} % extra features for tabular environment
\usepackage{amsmath}  % improve math presentation
\usepackage{graphicx} % takes care of graphic including machinery
\usepackage{geometry} % decreases margins
\usepackage[final]{hyperref} % adds hyper links inside the generated pdf file
\usepackage{booktabs}
\usepackage{subcaption}
\usepackage{tcolorbox}
\usepackage{fancyhdr}
\usepackage{authblk}
\usepackage[toc,page]{appendix}
\usepackage{parskip}
\usepackage{amssymb, amsmath} % Paquetes matemáticos de la American Mathematical Society
\usepackage{float}
\usepackage{setspace}
\usepackage{parskip}
\usepackage{multirow}
\usepackage[all]{xy}
\usepackage{tikz}
\usetikzlibrary{matrix}
\usetikzlibrary{calc}
\usetikzlibrary{fit}
%\usepackage{showframe}

\geometry{
    papersize = {216mm, 279.4mm},
    width = 20cm,
    height = 25cm,
    headsep = 5mm,
    head = 2.8cm,
    marginpar = 2mm,
    includeall,
}

\fancyhf{}
\renewcommand{\headrulewidth}{0pt}
\fancyhead[LO,LE]{
    \begin{minipage}{3cm}
        \includegraphics[width=0.7\textwidth]{Escudo.jpg}
    \end{minipage}
}
\fancyhead[RO,RE]{
    \textsf{
        Transformada de Fourier aplicado a una función Diente de Sierra no periódica\\
        Teoría de telecomunicaciones I, Grupo A12\\
        \date{\today}   
    }
}
\fancyfoot[C]{\thepage}

\pagestyle{fancy}

\hypersetup{
	colorlinks=true,       % false: boxed links; true: colored links
	linkcolor=black,        % color of internal links
	citecolor=black,        % color of links to bibliography
	filecolor=magenta,     % color of file links
	urlcolor=black         
}
\spanishdecimal{.}

\tcbuselibrary{theorems}
%++++++++++++++++++++++++++++++++++++++++
%Content

\raggedbottom

\addto\captionsspanish{
    \renewcommand\appendixname{Anexo}
    \renewcommand\appendixpagename{Anexos}
    }

\renewcommand{\tablename}{Tabla}

\renewcommand{\baselinestretch}{0.8} % Para indicar el tamaño del entrelineado

\usepackage[small,compact]{titlesec}
\titleformat{\subsection}[wrap]
    {\large\normalfont\fontseries{b}\selectfont\filright}
    {\thesubsection.}{.5em}{}
\titlespacing{\subsection}
    {12pc}{1.5ex plus .1ex minus .2ex}{1pc}
            
\titleformat{\section}[wrap]
    {\large\normalfont\fontseries{b}\selectfont\filright}
    {\thesection.}{.5em}{}
\titlespacing{\section}
    {12pc}{1.5ex plus .1ex minus .2ex}{1pc}

\setlength{\parskip}{1.5mm} % Modificar espacio entre párrafos

\renewcommand*{\Authsep}{ y }
\renewcommand*{\Authand}{ y }
\renewcommand*{\Authands}{, }
\renewcommand*{\Affilfont}{\normalsize}
%\renewcommand*{\Authfont}{\bfseries}    % make author names boldface    
\setlength{\affilsep}{-2mm}   % set the space between author and affiliation

\renewcommand\Authfont{\fontsize{12}{12}\selectfont} % Cambiar tamaño de letra autores
\renewcommand\Affilfont{\fontsize{9}{9}\itshape} % Cambiar tamaño de letra afiliaciones de autores

\title{
    \fontsize{26}{26}\selectfont 
    \textbf{Transformada de Fourier
    \vspace{-5mm}}}
\author[1]{Jefry Nicolás Chicaiza}
\author[2]{Jose Nicolás Zambrano}
\affil[1]{\url{jefryn@unicauca.edu.co}
    \vspace{-2mm}}
\affil[2]{\url{jnzambranob@unicauca.edu.co}
    \vspace{-5mm}}
\date{}

\bibliography{bibliografia}

\begin{document}
\maketitle
\vspace{-5mm}
\thispagestyle{fancy}
\section{Introducción}\label{intro}
    En el siguiente documento se desarrollará el informe del Trabajo 2 de la asignatura 
    Teoría de las telecomunicaciones 1. El trabajo presenta inicialmente el desarrollo 
    analítico de la Transformada de Fourier a la señal planteada, la cual es del tipo 
    "diente de sierra"  trasladado en el tiempo y no periódica.
    
    Iniciar con el desarrollo analítico es necesario debido a que para alcanzar los 
    resultados esperados en la simulación, se requiere conocer de antemano la función que 
    representa dicha Transformada de la señal, esto permitirá comprobar los resultados 
    obtenidos por medio de un algoritmo realizado en MATLAB.
    
    Adicionalmente, las comprobaciones que se plantearán en simulación se realizan a través 
    de la Transformada Rápida de Fourier (FFT, Fast Fourier Transform), que es una clase de 
    algoritmo computacional usado en el procesamiento de señales digitales para reducir en gran 
    medida el número de cálculos en el uso de la Transformada Discreta de Fourier (DFT) y su 
    inversa, hace de la DFT un procesamiento viable e indispensable \cite{Poularikas2007}.
    
    Cabe destacar que la finalidad principal de este documento será realizar una comprobación de 
    ¿Cómo se ve afectado el espectro en frecuencia al modificar ciertos parámetros de la señal?
    Para lo que se espera que la modificación de un parámetro que afecte directamente al espectro 
    de la magnitud, también afecte a su componente en fase.
    
    Finalmente, el documento cuenta con una sección para la validación de los diferentes procesos
    empleados y el planteamiento de un plan de pruebas para analizar los efectos que presentan
    algunas propiedades consideradas para la investigación de este documento, además de una sección
    de análisis de resultados y discusión, y seguido de la sección de conclusiones.
       
\section{Metodología}
    La metodología que se empleo para el desarrollo de la Transformada de Fourier a la señal planteada 
    se logró mediante la aplicación de los teoremas del material guía a este documento, inicialmente
    se realizó el cálculo analíticamente para considerar un resultado a partir de dicha teoría y
    poder comprobar el resultado en una simulación computacional. 
    
    Para realizar el proceso del cálculo de la transformada de la señal, es importante mencionar
    de donde surge el término de la Transformada de Fourier, surge a partir del estudio de señales
    periódicas que son representadas por las Series de Fourier, sin embargo, el estudio de las 
    señales se interrumpe por las señales aperiódicas por no cumplir con la condición de periodicidad
    \cite{Sauchelli2020}. Por tanto, se plantea que las señales no periódicas se pueden ver como 
    señales de periodos largos (que tienda al infinito), de esta forma la separación de sus 
    componentes espectrales es despreciable y el espectro de estas señales es continuo \cite{Silva2021}.
    
    Dado que la señal es no periódica, hablar de potencia promedio no seria adecuado debido a que 
    la magnitud tiende a cero cuando el tiempo tiende a infinito, en este caso la señal tiene una
    energía finita, la cual se define por una integral que puede existir o converger a un valor
    diferente de cero o indeterminado. Esta es la condición para que una señal pueda representarse
    en el dominio de la frecuencia utilizando la Transformada de Fourier \cite{Fabian}. Para la función de interés
    de este documento su valor de energía es:
    
    \begin{equation*}
        \epsilon_x=\int_{-\frac{1}{2}}^{\frac{7}{2}}\left|\frac{1}{4}t + \frac{1}{8}\right|^2 \mathrm{d}t = \frac{4}{3}
    \end{equation*}

    Después de comprobar que la señal tiene una energía definida y es posible denominarla como no
    periódica, se procedió a calcular la transformada de la señal resolviendo la integral 
    característica del teorema, antes de esto se considera la función que representa la grafica 
    de la señal de interés observada en la figura \ref{funcionAsiganda}.

    \begin{figure}[h!]
        \centering
        \selectlanguage{english}
        \begin{tikzpicture}[thick,scale=0.8, every node/.style={scale=0.8}]
	        \draw[dashed, gray!20](-1,0) grid (4,3.5);
		    \draw[line width = 0.5mm, ->](-1.5,0)--(4.5, 0) node[right]{$t$};
		    \draw[line width = 0.5mm, ->](0,-0.5)--(0,4) node[above]{$x(t)$};
		    \draw[black, line width = 0.3mm](-0.5,0)--(3.5,3) node[right]{$x(t) = \dfrac{1}{4}t +\dfrac{1}{8}$};
		    \draw[black, line width = 0.3mm](3.5,0)--(3.5,3) node[right]{};
            \draw[dashed, line width = 0.2mm](0,3)--(3.5,3) node[right]{};
		    \foreach \x in {-0.5,3.5}
		    \draw (\x, 1mm)--(\x, -1mm) node [below]{\scriptsize $\x$};
		    \foreach \y in {3}		
		    \draw (1mm, \y)--(-1mm, \y) node [left]{$1$};
        \end{tikzpicture}
        \selectlanguage{spanish}
        \caption{Gráfica de la señal asignada.}
        \label{funcionAsiganda}
    \end{figure}

    El resultado que se obtuvo de la transformada de esta señal con los intervalos observados en la
    figura \ref{funcionAsiganda} se aprecia en la ecuación \ref{resultadoFT} (el paso a paso del cálculo que se 
    realizó para esta transformada se puede encontrar anexo junto a este documento).
    
    \begin{equation}
        \tilde{x}(f) = \frac{e^{-j7pi f }(1 - e^{j8\pi f} + j8\pi f)}{16\pi^2 f^2}
        \label{resultadoFT}
    \end{equation} 
   
    Después de haber realizado el planteamiento de los fundamentos matemáticos requeridos para solucionar
    el problema, se procedió a realizar la planificación de la simulación requerida para verificar y analizar
    los escenarios planteados por el problema. 
    
    %Después de haber realizado estas validaciones para tener claridad acerca de lo que se 
    %observa en las gráficas, se dio inicio a la investigación que se plantea en este documento
    %con el análisis de algunos escenarios que se idearon en la simulación. las propiedades de 
    %la Transformada de Fourier que se usaron para el análisis fueron:
    %
    %\begin{itemize}
    %    \item Multiplicación por un escalar.
    %    \item Cambio de escala.
    %    \item Traslación en el tiempo.
    %    \item Traslación en frecuencia.
    %\end{itemize}
    
    Se puede identificar 6 objetivos clave a desarrollar con la simulación en MATLAB:
    
    \begin{enumerate}
        \item Cálculo y representación gráfica de la transformada de Fourier de la señal.
        \item Análisis de la Transformada versus FFT. 
        \item Análisis de los efectos sobre el espectro al multiplicar la señal por un escalar.
        \item Análisis de los efectos sobre el espectro al cambiar de escala del dominio de la señal.
        \item Análisis de los efectos sobre el espectro al trasladar la señal en el tiempo.
        \item Análisis de los efectos sobre el espectro al trasladar la señal en el dominio de la frecuencia.
    \end{enumerate}
    
    Después de realizar el análisis de los requerimientos anteriores, se plantea un \underline{esquema}
    \underline{general} para el desarrollo de la simulación en la figura \ref{diagramaGeneral}.
    
    \begin{figure}[h]
        \centerline{
            \xymatrix@ -1pc @ R = 10pt @ C = 0.4pt{
                % primera fila
                *++[F-,]\txt{1. Declaración de variables y espacios \\
                            vectoriales a utilizar.} \ar[d] & &
                \\
                % segunda fila
                *++[F-,]\txt{2. Calculo de la transformada de Fourier \\
                            de manera simbólica.} \ar[d] & &
                \\
                % tercer fila
                *++[F-,]\txt{3. Calculo de la transformada rápida de \\
                            Fourier.} \ar[d] & &
                \\
                % cuarta fila
                *++[F-,]\txt{4. Graficación de señales y espectros de \\
                            las Transformadas de Fourier} \ar[d] & &
                \\ 
                % quinta fila
                *++[F-,]\txt{5. TBD.} \ar[d] & &
                \\
                % sexta fila
                *++[F-,]\txt{6. TBD.}
            } 
        }
        \caption{Diagrama para el desarrollo de los escenarios.}
        \label{diagramaGeneral}
    \end{figure}
    
    Para la realización del Script de simulación se utilizó el paradigma de programación estructurada, 
    ya que el mismo permite agilidad en el desarrollo del código, así como también facilita el desarrollo
    de los planteamientos matemáticos necesarios para lograr los objetivos de la simulación.Adicionalmente
    la programación estructurada, brinda al observador del código una perspectiva secuencial en el 
    desarrollo del algoritmo de simulación, y, por lo tanto, un orden lógico y trazable de los resultados
    esperados en la ejecución de este.

    \subsection{Plan de pruebas}

    Retomando los 4 objetivos clave a desarrollar con la simulación, se plantea el siguiente plan de pruebas a realizar:
    
     \begin{figure}[H]
        \centerline{
            \xymatrix@ -1pc @ R = 10pt @ C = 0.4pt{
                % primera fila
                *++[F-,]\txt{1. Comparación de espectros: Simbólica vs FFT.} \ar[d] & &
                \\
                % segunda fila
                *++[F-,]\txt{2. Cambio de la amplitud en dominio temporal.} \ar[d] & &
                \\
                % tercer fila
                *++[F-,]\txt{3. Aumento y disminución del ancho de la \\
                            señal original.} \ar[d] & &
                \\
                % cuarta fila
                *++[F-,]\txt{4.Desplazamiento positivo en el dominio \\
                            del tiempo.} \ar[d] & &
                \\
                % quinta fila
                *++[F-,]\txt{5. Desplazamiento positivo y negativo del \\
                            espectro de magnitud.}
            }
        }
        \caption{Diagrama para el desarrollo de los escenarios.}
        \label{diagramaEscenarios}
    \end{figure}

    \section{Análisis de Resultados}
 
    Para la comprobación del resultado de la ecuación \ref{resultadoFT} se desarrollo un algoritmo 
    capas de calcular la integral característica de la Transformada de Fourier de tal manera que 
    el resultado obtenido sea fácil de comparar. En la siguiente figura se visualiza el resultado 
    que retorna el programa de simulación al ejecutar la integral.
    
    \begin{figure}[H]
        \centering 
        \centering
        \includegraphics[width=0.5\linewidth]{img/ResultadoEcuacionMATLAB.png}
        \caption{Resultado de la simulación de la Transformada de Fourier de la señal.}
        \label{resultadoEcuacionMATLAB}
    \end{figure} 
    
    Además de comprobar el resultado numérico de los cálculos, se implementó en el algoritmo un
    ciclo capaz de evaluar el resultado de la figura \ref{resultadoEcuacionMATLAB} para poder 
    ilustrar el espectro de la frecuencia de la transformada de la señal planteada.
     
    \begin{figure}[h]
        \centering 
        \begin{subfigure}[h]{0.49\linewidth}
            \includegraphics[width=\linewidth]{img/EMagnitudTF_ESimbolica.png}
            \caption{}
            \label{magnitudTeorica}
        \end{subfigure}
        \begin{subfigure}[h]{0.49\linewidth}
            \includegraphics[width=\linewidth]{img/EFaseTF_ESimbolica.png}
            \caption{}
            \label{faseTorica}
        \end{subfigure}
        \caption{Gráficas del espectro de frecuencia de los cálculos simulados. (a) Gráfica 
                espectro de magnitud, (b) Gráfica espectro de fase.}
        \label{espectroTeorico}
    \end{figure} 

    A partir de las figuras \ref{magnitudTeorica} y \ref{faseTorica} obtenidas en la simulación, se 
    realizaron comprobaciones con ayuda del algoritmo de la Transformada Rápida de Fourier (mencionada
    en la sección \ref{intro}).
    
    El proceso de comprobación de las gráficas inicialmente requirió de una buena
    implementación del algoritmo FFT para realizar una validez al aplicar la propiedad de dualidad que
    manifiesta que $V(t)\rightarrow v(-f)$, donde la transformada en el dominio del tiempo su transformada
    será la señal original en el dominio de la frecuencia (en la simulación se realizaron las
    comprobaciones),  por lo que se valida que la señal en el tiempo es discreta y en la 
    frecuencia es de tipo periódico como nos plantea la teoría de la Transformada de Fourier,
    las gráficas obtenidas por el cálculo de la FFT se observan en la figura \ref{espectroFFT}.
    
    \begin{figure}[h]
        \centering 
        \begin{subfigure}[h]{0.49\linewidth}
            \includegraphics[width=\linewidth]{img/EMagnitudTF_FFT.png}
            \label{magnitudFFT}
            \caption{}
        \end{subfigure}
        \begin{subfigure}[h]{0.49\linewidth}
            \includegraphics[width=\linewidth]{img/EFaseTF_FFT.png}
            \label{faseFFT}
            \caption{}
        \end{subfigure}
        \caption{Gráficas del espectro de frecuencia de FFT. (a) Gráfica 
                espectro de magnitud, (b) Gráfica espectro de fase.}
        \label{espectroFFT}
    \end{figure} 
    
    
    
    \section{Conclusiones}

%\appendix
%\section{Cálculo teórico}

    %Realización de los cálculos matemáticos de la Transformada de Fourier de la señal 
    %asignada, la formula empleada para el cálculo de la transformada es la siguiente:

    %\begin{equation}
        %\tilde{x}(f)=\int_{-\infty}^{\infty}x(t)e^{-j2\pi ft} \mathrm{d}t
        %\label{transformadaFourier}
    %\end{equation}

    %Recordando que del primer trabajo, la función correspondiente a la pendiente de la 
    %gráfica se expresa de la siguiente manera:

    %\begin{equation}
        %x(t)=\frac{1}{4} t + \frac{1}{8}
        %\label{funcionLineal}
    %\end{equation}

    %Esta función lineal se encuentra limitada por un intervalo de duración $4$ segundos, como valor
    %mínimo se tiene $-\frac{1}{2}$ y valor máximo $\frac{7}{2}$, lo que hace que esta 
    %función se vea como un "diente de sierra", por tanto su función más representativa o
    %su intervalo descriptivo es el siguiente:
    
    %\begin{equation}
        %\begin{split}
            %x(t)&=\left(\frac{1}{4} t + \frac{1}{8}\right)rect\left(\frac{t}{4} + \frac{3}{8}\right) \\
                %&=  \left\lbrace  \begin{array}{ll}
                                    %\frac{1}{4} t + \frac{1}{8}; & -\frac{1}{2} \leq t \leq \frac{7}{2} \\
                                    %0; & p.o.c.
                                %\end{array}
                    %\right.      
        %\end{split}
        %\label{dienteSierra}
    %\end{equation}

    %El proceso para obtener la Transformada de Fourier de la ecuación \ref{dienteSierra} 
    %se realiza a continuación, donde el valor $x(t)$ de la ecuación \ref{transformadaFourier} será la 
    %ecuación \ref{funcionLineal}, y el valor de los intervalos de la integral serán los que 
    %limitan la función lineal, como se menciono anteriormente \cite{silviaRB}: 

    %\begin{equation*}
        %\begin{split}
            %\tilde{x}(f)& = \int_{\frac{3}{2} -2}^{\frac{3}{2} +2}\left(\frac{1}{4} t+\frac{1}{8} \right) e^{-j2\pi ft}\mathrm{d}t \\
                        %& = \frac{1}{4} \int_{\frac{3}{2}-2}^{\frac{3}{2}+2}t e^{-j2\pi ft}\mathrm{d}t + \frac{1}{8} \int_{\frac{3}{2} -2}^{\frac{2}{3} +2} e^{-j2\pi ft}\mathrm{d}t \\
                        %& = e^{-j2\pi ft}\left(\frac{jt}{8\pi f}  + \frac{1}{16\pi ^{2} f^2}  + \frac{j}{16\pi f} \right)  \Big|_{\frac{3}{2} -2}^{\frac{3}{2} +2} \\
                        %& = \frac{je^{-j3\pi ft } e^{-j4\pi ft}}{8\pi f} \left[\frac{1}{j2\pi f} + 4 
                          %- e^{j8\pi ft} \left(\frac{1}{j2\pi f} \right)\right] \\
                        %& = \left[\left(\frac{1}{16\pi^2 f^2}+\frac{j}{2\pi f}\right)e^{-j4\pi f}-\frac{e^{j4\pi f}}{16\pi^2 f^2}\right] e^{-j3\pi f} \\
        %\end{split}
    %\end{equation*}  
  

    %\begin{equation}
        %\tcboxmath[colback=gray!25!white,colframe=black, title=Transformada de Fourier]{
            %\tilde{x}(f) = \frac{e^{-j7pi f }(1 - e^{j8\pi f} + j8\pi f)}{16\pi^2 f^2}
        %}
    %\end{equation}

\printbibliography[title={Bibliografía}]

\end{document}
