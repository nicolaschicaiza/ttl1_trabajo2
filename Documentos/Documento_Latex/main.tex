%Preamble
\documentclass[11pt,a4paper,twocolumn]{article}
\brokenpenalty=10000 
%\hyphenpenalty=10000 
\usepackage[spanish]{babel}
\usepackage[utf8]{inputenc}
\usepackage{times}
\usepackage[backend=biber,style=ieee]{biblatex}
\usepackage[T1]{fontenc}
\usepackage{cancel}
\usepackage{tabularx} % extra features for tabular environment
\usepackage{amsmath}  % improve math presentation
\usepackage{graphicx} % takes care of graphic including machinery
\usepackage{geometry} % decreases margins
\usepackage[final]{hyperref} % adds hyper links inside the generated pdf file
\usepackage{booktabs}
\usepackage{subcaption}
\usepackage{tcolorbox}
\usepackage{fancyhdr}
\usepackage{authblk}
\usepackage[toc,page]{appendix}
\usepackage{parskip}
\usepackage{amssymb, amsmath} % Paquetes matemáticos de la American Mathematical Society
\usepackage{float}
\usepackage{multirow}
\usepackage[all]{xy}
\usepackage{tikz}
\usetikzlibrary{matrix}
\usetikzlibrary{calc}
\usetikzlibrary{fit}
%\usepackage{showframe}

\geometry{
    papersize = {216mm, 279mm},
    width = 20cm,
    height = 25cm,
    headsep = 5mm,
    head = 2.8cm,
    marginpar = 2mm,
    includeall,
}

\fancyhf{}
\renewcommand{\headrulewidth}{0pt}
\fancyhead[LO,LE]{
    \begin{minipage}{3cm}
        \includegraphics[width=0.7\textwidth]{Escudo.jpg}
    \end{minipage}
}
\fancyhead[RO,RE]{
    \textsf{
        Transformada de Fourier aplicado a una función Diente de Sierra no periódica\\
        Teoría de telecomunicaciones I, Grupo A12\\
        \date{\today}   
    }
}
\fancyfoot[C]{\thepage}

\pagestyle{fancy}

\hypersetup{
	colorlinks=true,       % false: boxed links; true: colored links
	linkcolor=black,        % color of internal links
	citecolor=black,        % color of links to bibliography
	filecolor=magenta,     % color of file links
	urlcolor=blue         
}
\spanishdecimal{.}

\tcbuselibrary{theorems}
%++++++++++++++++++++++++++++++++++++++++
%Content

\addto\captionsspanish{
    \renewcommand\appendixname{Anexo}
    \renewcommand\appendixpagename{Anexos}
    }

\renewcommand{\tablename}{Tabla}

\renewcommand{\baselinestretch}{0.8} % Para indicar el tamaño del entrelineado

\usepackage[small,compact]{titlesec}
\titleformat{\subsubsection}[runin]
            {\normalfont\it}
            {\thesubsubsection}{0.5em}{}[:]

\renewcommand*{\Authsep}{ y }
\renewcommand*{\Authand}{ y }
\renewcommand*{\Authands}{, }
\renewcommand*{\Affilfont}{\normalsize}
%\renewcommand*{\Authfont}{\bfseries}    % make author names boldface    
\setlength{\affilsep}{0.5em}   % set the space between author and affiliation

\title{
    \fontsize{26}{26}\selectfont 
    \textbf{Transformada de Fourier}}

\author[1]{Jefry Nicolás Chicaiza}
\author[2]{Jose Nicolás Zambrano}
\affil[1]{jefryn@unicauca.edu.co}
\affil[2]{jnzambranob@unicauca.edu.co}
\date{}

\bibliography{bibliografia}

\begin{document}
\maketitle
\thispagestyle{fancy}
\section{Introducción}
    En el siguiente documento se desarrollará el informe del Trabajo 2 de la asignatura 
    Teoría de las telecomunicaciones 1. El trabajo presenta inicialmente el desarrollo 
    analítico de la Transformada de Fourier a la señal planteada, la cual es del tipo 
    "diente de sierra"  trasladado en el tiempo y no periódica.
    
    Iniciar con el desarrollo analítico es necesario debido a que para alcanzar los 
    resultados esperados en la simulación, se requiere conocer de antemano la función que 
    representa dicha Transformada de la señal, esto permitirá comprobar los resultados 
    obtenidos por medio de un algoritmo realizado en MATLAB.
    
    Adicionalmente, las comprobaciones que se plantearán en simulación se realizan a través 
    de la Transformada Rápida de Fourier (FFT, Fast Fourier Transform), que es una clase de 
    algoritmo computacional usado en el procesamiento de señales digitales para reducir en gran 
    medida el número de cálculos en el uso de la Transformada Discreta de Fourier (DFT) y su 
    inversa, hace de la DFT un procesamiento viable e indispensable \cite{Poularikas2007}.

       
\section{Metodología}
    La metodología que se empleo para el desarrollo de la Transformada de Fourier a la señal planteada 
    se logró mediante la aplicación de los teoremas del material guía a este documento, inicialmente
    se realizó el cálculo analíticamente para considerar un resultado a partir de dicha teoría y
    poder comprobar el resultado en una simulación computacional. 
    
    Para realizar el proceso del cálculo de la transformada de la señal es importante mencionar
    de donde surge el término de la Transformada de Fourier, surge a partir del estudio de señales
    periódicas que son representadas por las Series de Fourier, sin embargo, el estudio de las 
    señales se interrumpe por las señales aperiódicas por no cumplir con la condición de periodicidad \cite{Sauchelli2020}.
    Por tanto, se plantea que las señales no periódicas se pueden ver como señales de periodos 
    largos (que tienda al infinito), de esta forma la separación de sus componentes espectrales es
    despreciable y el espectro de estas señales es continuo \cite{Silva2021}.
    
    Dado que la señal es no periódica, hablar de potencia promedio no seria adecuado debido a que 
    la magnitud tiende a cero cuando el tiempo tiende a infinito, en este caso la señal tiene una
    energía finita, la cual se define por una integral que puede existir o converger a un valor
    diferente de cero o indeterminado. Esta es la condición para que una señal pueda representarse
    en el dominio de la frecuencia utilizando la Transformada de Fourier \cite{Fabian}. Para la función de interés
    de este documento su valor de energía es:
    
    \begin{equation*}
       \epsilon_x=\int_{-\frac{1}{2}}^{\frac{7}{2}}\left|\frac{1}{4}t + \frac{1}{8}\right|^2 \mathrm{d}t = \frac{4}{3}
    \end{equation*}

    Después de comprobar que la señal tiene una energía definida y es posible denominarla como no
    periódica, se procedió a calcular la transformada de la señal resolviendo la integral 
    característica del teorema, antes de esto se considera la función que representa la gráfica
    de esta señal de interés:

    \begin{figure}[H]
        \centering
        \selectlanguage{english}
        \begin{tikzpicture}
	        \draw[dashed, gray!20](-1,0) grid (4,3.5);
		    \draw[line width = 0.5mm, ->](-1.5,0)--(4.5, 0) node[right]{$t$};
		    \draw[line width = 0.5mm, ->](0,-0.5)--(0,4) node[above]{$x(t)$};
		    \draw[black, line width = 0.3mm](-0.5,0)--(3.5,3) node[right]{$x(t) = \dfrac{1}{4}t +\dfrac{1}{8}$};
		    \draw[black, line width = 0.3mm](3.5,0)--(3.5,3) node[right]{};
            \draw[dashed, line width = 0.2mm](0,3)--(3.5,3) node[right]{};
		    \foreach \x in {-0.5,3.5}
		    \draw (\x, 1mm)--(\x, -1mm) node [below]{\scriptsize $\x$};
		    \foreach \y in {3}		
		    \draw (1mm, \y)--(-1mm, \y) node [left]{$1$};
        \end{tikzpicture}
        \selectlanguage{spanish}
        \caption{Gráfica de la señal tipo "diente de sierra" no periódica.}
        \label{funcionAsiganda}
    \end{figure}

    El resultado que se obtuvo de la transformada de esta señal con los intervalos observados en la
    figura \ref{funcionAsiganda} se aprecia a continuación (el paso a paso de el cálculo que se 
    realizó para esta transformada se puede encontrar en la sección de anexos de este documento):
    
    \begin{equation}
        \tilde{x}(f) = \frac{e^{-j7pi f }(1 - e^{j8\pi f} + j8\pi f)}{16\pi^2 f^2}
    \end{equation} 

    

    \subsection{Plan de pruebas}

\section{Análisis de Resultados}
\section{Conclusiones}

\appendix
\section{Cálculo teórico}

Realización de los cálculos matemáticos de la Transformada de Fourier de la señal 
    asignada, la formula empleada para el cálculo de la transformada es la siguiente:

    \begin{equation}
        \tilde{x}(f)=\int_{-\infty}^{\infty}x(t)e^{-j2\pi ft} \mathrm{d}t
        \label{transformadaFourier}
    \end{equation}

    Recordando que del primer trabajo, la función correspondiente a la pendiente de la 
    gráfica se expresa de la siguiente manera:

    \begin{equation}
        x(t)=\frac{1}{4} t + \frac{1}{8}
        \label{funcionLineal}
    \end{equation}

    Esta función lineal se encuentra limitada por un intervalo de duración $4$ segundos, como valor
    mínimo se tiene $-\frac{1}{2}$ y valor máximo $\frac{7}{2}$, lo que hace que esta 
    función se vea como un "diente de sierra", por tanto su función más representativa o
    su intervalo descriptivo es el siguiente:
    
    \begin{equation}
        \begin{split}
            x(t)&=\left(\frac{1}{4} t + \frac{1}{8}\right)rect\left(\frac{t}{4} + \frac{3}{8}\right) \\
                &=  \left\lbrace  \begin{array}{ll}
                                    \frac{1}{4} t + \frac{1}{8}; & -\frac{1}{2} \leq t \leq \frac{7}{2} \\
                                    0; & p.o.c.
                                \end{array}
                    \right.      
        \end{split}
        \label{dienteSierra}
    \end{equation}

    El proceso para obtener la Transformada de Fourier de la ecuación \ref{dienteSierra} 
    se realiza a continuación, donde el valor $x(t)$ de la ecuación \ref{transformadaFourier} será la 
    ecuación \ref{funcionLineal}, y el valor de los intervalos de la integral serán los que 
    limitan la función lineal, como se menciono anteriormente: \cite{silviaRB}

    \begin{equation*}
        \begin{split}
            \tilde{x}(f)& = \int_{\frac{3}{2} -2}^{\frac{3}{2} +2}\left(\frac{1}{4} t+\frac{1}{8} \right) e^{-j2\pi ft}\mathrm{d}t \\
                        & = \frac{1}{4} \int_{\frac{3}{2}-2}^{\frac{3}{2}+2}t e^{-j2\pi ft}\mathrm{d}t + \frac{1}{8} \int_{\frac{3}{2} -2}^{\frac{2}{3} +2} e^{-j2\pi ft}\mathrm{d}t \\
                        & = e^{-j2\pi ft}\left(\frac{jt}{8\pi f}  + \frac{1}{16\pi ^{2} f^2}  + \frac{j}{16\pi f} \right)  \Big|_{\frac{3}{2} -2}^{\frac{3}{2} +2} \\
                        & = \frac{je^{-j3\pi ft } e^{-j4\pi ft}}{8\pi f} \left[\frac{1}{j2\pi f} + 4 
                          - e^{j8\pi ft} \left(\frac{1}{j2\pi f} \right)\right] \\
                        & = \left[\left(\frac{1}{16\pi^2 f^2}+\frac{j}{2\pi f}\right)e^{-j4\pi f}-\frac{e^{j4\pi f}}{16\pi^2 f^2}\right] e^{-j3\pi f} \\
        \end{split}
    \end{equation*}  
  

    \begin{equation}
        \tcboxmath[colback=gray!25!white,colframe=black, title=Transformada de Fourier]{
            \tilde{x}(f) = \frac{e^{-j7pi f }(1 - e^{j8\pi f} + j8\pi f)}{16\pi^2 f^2}
        }
    \end{equation}

\printbibliography[title={Bibliografía}]

\end{document}
