%Preamble
\documentclass[11pt,a4paper,twocolumn]{article}
\brokenpenalty=10000 
\hyphenpenalty=10000 
\usepackage[spanish]{babel}
\usepackage[utf8]{inputenc}
\usepackage{times}
\usepackage[backend=biber,style=ieee]{biblatex}
\usepackage[T1]{fontenc}
\usepackage{cancel}
\usepackage{tabularx} % extra features for tabular environment
\usepackage{amsmath}  % improve math presentation
\usepackage{graphicx} % takes care of graphic including machinery
\usepackage{geometry} % decreases margins
\usepackage[final]{hyperref} % adds hyper links inside the generated pdf file
\usepackage{booktabs}
\usepackage{subcaption}
\usepackage{tcolorbox}
\usepackage{fancyhdr}
\usepackage{authblk}
\usepackage{parskip}
\usepackage{amssymb, amsmath} % Paquetes matemáticos de la American Mathematical Society
\usepackage{float}
\usepackage{multirow}
\usepackage[all]{xy}
\usepackage{tikz}
\usetikzlibrary{matrix}
\usetikzlibrary{calc}
\usetikzlibrary{fit}
%\usepackage{showframe}

\geometry{
    papersize = {216mm, 279mm},
    width = 20cm,
    height = 25cm,
    headsep = 5mm,
    head = 2.8cm,
    marginpar = 2mm,
    includeall,
}

\fancyhf{}
\renewcommand{\headrulewidth}{0pt}
\fancyhead[LO,LE]{
    \begin{minipage}{3cm}
        \includegraphics[width=0.7\textwidth]{Escudo.jpg}
    \end{minipage}
}
\fancyhead[RO,RE]{
    \textsf{
        Transformada de Fourier aplicado a una función Diente de Sierra no periódica\\
        Teoría de telecomunicaciones I, Grupo A12\\
        \date{\today}   
    }
}
\fancyfoot[C]{\thepage}

\pagestyle{fancy}

\hypersetup{
	colorlinks=true,       % false: boxed links; true: colored links
	linkcolor=black,        % color of internal links
	citecolor=black,        % color of links to bibliography
	filecolor=magenta,     % color of file links
	urlcolor=blue         
}
\spanishdecimal{.}

%++++++++++++++++++++++++++++++++++++++++
%Content
\usepackage[small,compact]{titlesec}
\titleformat{\subsubsection}[runin]
            {\normalfont\it}
            {\thesubsubsection}{0.5em}{}[:]

\renewcommand*{\Authsep}{ y }
\renewcommand*{\Authand}{ y }
\renewcommand*{\Authands}{, }
\renewcommand*{\Affilfont}{\normalsize}
%\renewcommand*{\Authfont}{\bfseries}    % make author names boldface    
\setlength{\affilsep}{0.5em}   % set the space between author and affiliation

\title{
    \fontsize{26}{26}\selectfont 
    \textbf{Transformada de Fourier}}

\author[1]{Jefry Nicolás Chicaiza}
\author[2]{Jose Nicolás Zambrano}
\affil[1]{jefryn@unicauca.edu.co}
\affil[2]{jnzambranob@unicauca.edu.co}
\date{}

\bibliography{bibliografia}

\begin{document}
\maketitle
\thispagestyle{fancy}
\section{Introducción}
    En el siguiente documento se desarrollará el informe del Trabajo 2 de la asignatura 
    Teoría de las telecomunicaciones 1. El trabajo presenta inicialmente el desarrollo 
    analítico de la Transformada de Fourier a la señal planteada, la cual es del tipo 
    "diente de sierra"  trasladado en el tiempo y no periódica.
    
    Iniciar con el desarrollo analítico es necesario debido a que para alcanzar los 
    resultados esperados en la simulación, se requiere conocer de antemano la función que 
    representa dicha Transformada de la señal, esto permitirá comprobar los resultados 
    obtenidos por medio de un algoritmo realizado en MATLAB.\cite{silviaRB}
    
    Adicionalmente, las comprobaciones que se plantearán en simulación se realizan a través 
    de la Transformada Rápida de Fourier (FFT, Fast Fourier Transform), que es una clase de 
    algoritmo computacional usado en el procesamiento de señales digitales para reducir en gran 
    medida el número de cálculos en el uso de la Transformada Discreta de Fourier (DFT) y su 
    inversa, hace de la DFT un procesamiento viable e indispensable.\cite{Poularikas2007}

       
\section{Metodología}
    

    \begin{figure}[H]
        \centering
        \selectlanguage{english}
        \begin{tikzpicture}
	        \draw[dashed, gray!20](-1,0) grid (4,3.5);
		    \draw[line width = 0.5mm, ->](-1.5,0)--(4.5, 0) node[right]{$t$};
		    \draw[line width = 0.5mm, ->](0,-0.5)--(0,4) node[above]{$x(t)$};
		    \draw[black, line width = 0.3mm](-0.5,0)--(3.5,3) node[right]{$x(t) = \dfrac{1}{4}t +\dfrac{1}{8}$};
		    \draw[black, line width = 0.3mm](3.5,0)--(3.5,3) node[right]{};
            \draw[dashed, line width = 0.2mm](0,3)--(3.5,3) node[right]{};
		    \foreach \x in {-0.5,3.5}
		    \draw (\x, 1mm)--(\x, -1mm) node [below]{\scriptsize $\x$};
		    \foreach \y in {3}		
		    \draw (1mm, \y)--(-1mm, \y) node [left]{$1$};
        \end{tikzpicture}
        \selectlanguage{spanish}
        \caption{Gráfica de la señal tipo "diente de sierra" no periódica.}
        \label{funcionAsiganda}
    \end{figure}

    \subsection{Plan de pruebas}

\section{Análisis de Resultados}



\section{Conclusiones}
\printbibliography[title={Bibliografía}]

\end{document}
