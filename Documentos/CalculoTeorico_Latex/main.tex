%Preamble
\documentclass[11pt, twocolumn]{article}
\usepackage[spanish]{babel}
\usepackage[utf8]{inputenc}
\usepackage{times}
\usepackage[backend=biber,style=ieee]{biblatex}
\usepackage[T1]{fontenc}
\usepackage{cancel}
\usepackage{tabularx} % extra features for tabular environment
\usepackage{amsmath}  % improve math presentation
\usepackage{graphicx} % takes care of graphic including machinery
\usepackage{geometry} % decreases margins
\usepackage[final]{hyperref} % adds hyper links inside the generated pdf file
\usepackage{booktabs}
\usepackage{subcaption}
\usepackage{tcolorbox}
\usepackage{fancyhdr}
\usepackage{authblk}
\usepackage{parskip}
\usepackage{amssymb, amsmath} % Paquetes matemáticos de la American Mathematical Society
\usepackage{float}
\usepackage{multirow}
\usepackage[all]{xy}
\usepackage{tikz}
\usetikzlibrary{matrix}
\usetikzlibrary{calc}
\usetikzlibrary{fit}
%\usepackage{showframe}

\geometry{
    papersize = {216mm, 279mm},
    width = 18.6cm,
    height = 25cm,
    headsep = 5mm,
    head = 2.8cm,
    marginpar = 5mm,
    includeall,
}

\fancyhf{}
\renewcommand{\headrulewidth}{0pt}
\fancyhead[LO,LE]{
    \begin{minipage}{3cm}
        \includegraphics[width=0.7\textwidth]{Escudo.jpg}
    \end{minipage}
}
\fancyhead[RO,RE]{
    \textsf{
        \fontsize{11}{11}\selectfont 
        Transformada de Fourier aplicado a una función no periodia Diente de Sierra\\
        Teoría de telecomunicaciones I, Grupo A12\\
        \date{\today}   
    }
}
\fancyfoot[C]{\thepage}

\pagestyle{fancy}

\hypersetup{
	colorlinks=true,       % false: boxed links; true: colored links
	linkcolor=black,        % color of internal links
	citecolor=black,        % color of links to bibliography
	filecolor=magenta,     % color of file links
	urlcolor=blue         
}
\spanishdecimal{.}

\tcbuselibrary{theorems}
%++++++++++++++++++++++++++++++++++++++++
%Content

\usepackage[small,compact]{titlesec}
\titleformat{\subsubsection}[runin]
            {\normalfont\it}
            {\thesubsubsection}{0.5em}{}[:]

\renewcommand*{\Authsep}{ y }
\renewcommand*{\Authand}{ y }
\renewcommand*{\Authands}{, }
\renewcommand*{\Affilfont}{\normalsize}
%\renewcommand*{\Authfont}{\bfseries}    % make author names boldface    
\setlength{\affilsep}{0.5em}   % set the space between author and affiliation


\title{
    \fontsize{22}{22}\selectfont 
    \textbf{Trabajo 2 -- Teoría de las telecomunicaciones I} \\
    \fontsize{16}{16}\selectfont 
    Tema: Transformada de Fourier
}

\author[1]{Jefry Nicolás Chicaiza}
\author[2]{Jose Nicolás Zambrano}
\affil[1]{jefryn@unicauca.edu.co}
\affil[2]{jnzambranob@unicauca.edu.co}
\date{}

\bibliography{bibliografia.bib}

\begin{document}
\maketitle
\thispagestyle{fancy}

\section*{Ejercicio}
Asumiendo que la señal de la figura no es periódica calcule su Transformada de Fourier. 

\begin{figure}[H]
    \centering
    \selectlanguage{english}
    \begin{tikzpicture}
	    \draw[dashed, gray!20](-1,0) grid (4,3.5);
		\draw[line width = 0.5mm, ->](-1.5,0)--(4.5, 0) node[right]{$t$};
		\draw[line width = 0.5mm, ->](0,-0.5)--(0,4) node[above]{$x(t)$};
		\draw[black, line width = 0.3mm](-0.5,0)--(3.5,3) node[right]{$x(t) = \dfrac{1}{4}t +\dfrac{1}{8}$};
		\draw[black, line width = 0.3mm](3.5,0)--(3.5,3) node[right]{};
        \draw[dashed, line width = 0.2mm](0,3)--(3.5,3) node[right]{};
		\foreach \x in {-0.5,1.5,3.5}
		\draw (\x, 1mm)--(\x, -1mm) node [below]{\scriptsize $\x$};
		\foreach \y in {3}		
		\draw (1mm, \y)--(-1mm, \y) node [left]{$1$};
    \end{tikzpicture}
    \selectlanguage{spanish}
\end{figure}

Después compruebe los resultados obtenidos en simulación al aplicar la Transformada Rápida 
Fourier (FFT, Fast Fourier Transform).

Para interpretar los resultados de la FFT se sugiere analizar la dualidad en tiempo y 
frecuencia, ante las características de periódico y discreto. Adicionalmente, es posible 
simular el cálculo del espectro, con el fin de contrastar los resultados obtenidos con 
el de la FFT, el cual corresponde al algoritmo computacionalmente más eficiente.

Una vez se tenga claridad con respecto a las gráficas del espectro de la señal, se 
deben analizar los efectos que tienen, sobre el espectro de la señal, las propiedades 
de:
\begin{itemize}
    \item Multiplicación por un escalar.
    \item Cambio de escala.
    \item Traslación en el tiempo.
    \item Traslación en frecuencia.
\end{itemize}

\section*{Solución}
    Realización de los cálculos matemáticos de la Transformada de Fourier de la señal 
    asignada, la formula empleada para el cálculo de la transformada es la siguiente:

    \begin{equation}
        \tilde{x}(f)=\int_{-\infty}^{\infty}x(t)e^{-j2\pi ft} \mathrm{d}t
        \label{transformadaFourier}
    \end{equation}

    Recordando que del primer trabajo, la función correspondiente a la pendiente de la 
    gráfica se expresa de la siguiente manera:

    \begin{equation}
        x(t)=\frac{1}{4} t + \frac{1}{8}
        \label{funcionLineal}
    \end{equation}

    Esta función lineal se encuentra limitada por un intervalo de duración $4$ segundos, como valor
    mínimo se tiene $-\frac{1}{2}$ y valor máximo $\frac{7}{2}$, lo que hace que esta 
    función se vea como un "diente de sierra", por tanto su función más representativa o
    su intervalo descriptivo es el siguiente:
    
    \onecolumn
    \begin{equation}
        x(t)=\left(\frac{1}{4} t + \frac{1}{8}\right)rect\left(\frac{t}{4} + \frac{3}{8}\right)
         =\left\lbrace  \begin{array}{ll}
                            \frac{1}{4} t + \frac{1}{8}; & -\frac{1}{2} \leq t \leq \frac{7}{2} \\
                            0; & p.o.c.
                        \end{array}
             \right.
        \label{dienteSierra}
    \end{equation}

    El proceso para obtener la Transformada de Fourier de la ecuación \ref{dienteSierra} 
    se realiza a continuación, donde el valor $x(t)$ de la ecuación \ref{transformadaFourier} será la 
    ecuación \ref{funcionLineal}, y el valor de los intervalos de la integral serán los que 
    limitan la función lineal, como se menciono anteriormente: \cite{silviaRB}

    \begin{equation*}
        \begin{split}
            \tilde{x}(f)& = \int_{\frac{3}{2} -2}^{\frac{3}{2} +2}\left(\frac{1}{4} t+\frac{1}{8} \right) e^{-j2\pi ft}\mathrm{d}t \\
                        & = \frac{1}{4} \int_{\frac{3}{2}-2}^{\frac{3}{2}+2}t e^{-j2\pi ft}\mathrm{d}t + \frac{1}{8} \int_{\frac{3}{2} -2}^{\frac{2}{3} +2} e^{-j2\pi ft}\mathrm{d}t \\
                        & = \frac{jt}{8\pi f} e^{-j2\pi ft} + \frac{1}{16\pi ^{2} f^2} e^{-j2\pi ft} + \frac{j}{16\pi f} e^{-j2\pi ft} \Big|_{\frac{3}{2} -2}^{\frac{3}{2} +2} \\
                        & = \frac{je^{-j2\pi ft\frac{3}{2}}}{8\pi f} \left\{e^{-j4\pi ft} \left[ \left(\frac{3}{2} + 2\right) + \frac{1}{j2\pi f} + \frac{1}{2}\right] 
                          - e^{j4\pi ft} \left[\left(\frac{3}{2} - 2\right) + \frac{1}{j2\pi f} + \frac{1}{2}\right]\right\} \\
                        & = \frac{e^{-j3\pi f}}{4\pi f} \left\{e^{-j4\pi f}\left(\frac{j3}{4}+j+\frac{j}{4}+\frac{1}{4\pi f}\right)   
                          - e^{j4\pi f} \left(\cancel{\frac{j3}{4}}-\cancel{j}+\cancel{\frac{j}{4}}+\frac{1}{4\pi f}\right)\right\} \\
                        & = \left[\left(\frac{1}{16\pi^2 f^2}+\frac{j}{2\pi f}\right)e^{-j4\pi f}-\frac{e^{j4\pi f}}{16\pi^2 f^2}\right] e^{-j3\pi f} \\
        \end{split}
    \end{equation*}  

    \begin{equation}
        \tcboxmath[colback=gray!25!white,colframe=black, title=Transformada de Fourier]{
            \tilde{x}(f) = \frac{e^{-j7pi f }(1 - e^{j8\pi f} + j8\pi f)}{16\pi^2 f^2}
        }
    \end{equation}

\printbibliography[title={Bibliografía}]
\nocite{Silva2021}

\end{document}